\section{Product Data}
\paragraph{}Each time \app{} generates two XML-Configuration files. One for test configuration and another one for the tester's username and password.

\subsection{Configuration data}
\begin{description}
\item[/D00000/ telesales-buying-agent-batch]\hfill \\ Consists from  telesales-urls (\textbf{D01000}), customers (\textbf{D02000}) and orders (\textbf{D03000}).
\item[/D01000/ telesales-urls]\hfill \\ Consists from a list of telesales-url (\textbf{D01100}).
\item[/D01100/ telesales-url]\hfill \\ Contains country (\textbf{D01110}) and URL as a value (\textbf{D01120}).
\item[/D01110/ country]\hfill \\ Tester have to choose a country from the list of countries for testing. The chosen one will be added automatically into the all needed places in the XML-File.
\item[/D01120/ value (URL)]\hfill \\ Tester have to enter the URL for testing.

\item[/D02000/ customers]\hfill \\ Consists from a list of customer (\textbf{D02100}).
\item[/D02100/ customer]\hfill \\ Contains country (\textbf{D01110}) and customer-number (\textbf{D02110}).
\item[/D02110/ customer-number]\hfill \\ Contains customer-number as a value. This is a number of a test customer account.

\item[/D03000/ orders]\hfill \\ Blablabla


\item[/D020/ BlaBlaBla]\hfill \\ Blablabla
\item[/D030/ BlaBlaBla]\hfill \\ Blablabla
\end{description}

\begin{landscape}
\paragraph{Example:}
\begin{verbatim}
<?xml version="1.0"?>
<telesales-buying-agent-batch>
  <telesales-urls>
    <telesales-url country="DE">https://hosting-ts-de.ts-host.gem1.sales.united.domain:9613/</telesales-url>
    <telesales-url country="ES">https://hosting-ts-es.ts-host.gem1.sales.united.domain:9613/</telesales-url>
    <telesales-url country="FR">https://hosting-ts-fr.ts-host.gem1.sales.united.domain:9613/</telesales-url>
    …
  </telesales-urls>
  <customers>
    <customer country="DE">
      <customer-number>7156019</customer-number>
    </customer>
    <customer country="FR">
      <customer-number>207958780</customer-number>
    </customer>
    …
  </customers>
  <orders>
    <order>
      <country>DE</country>
      <tariff>tariff-basic</tariff>
      <tariff-campaign-control>tariff-toggle</tariff-campaign-control>
      <tariff-addons>
        <tariff-addon id="presales.articles.slot-eshop-addon">opt-addon-eshop-basic</tariff-addon>
        <tariff-addon id="presales.articles.slot-sitelock-basic-addon">opt-addon-sitelock-basic</tariff-addon>
        <tariff-addon id="presales.articles.slot-seotool-addon">opt-addon-seotool</tariff-addon>
        …
      </tariff-addons>
      <domain>deinedomain{TIMESTAMP}.de</domain> <!-- {TIMESTAMP} will be replaced by yyyyMMddHHmmss -->
      <domain-bundle>false</domain-bundle>
    </order>
    <order>
      <country>FR</country>
      <tariff>tariff-basic</tariff>
      <tariff-campaign-control>tariff-toggle</tariff-campaign-control>
      <tariff-addons />
      <domain>votredomain{TIMESTAMP}.fr</domain> <!-- {TIMESTAMP} will be replaced by yyyyMMddHHmmss -->
      <domain-bundle>true</domain-bundle>
    </order>
  </orders>
</telesales-buying-agent-batch>
\end{verbatim}
\end{landscape}

\subsection{Login data}
\paragraph{}This data will be used as a login data of the testers.
\begin{description}
\item[/DC010/ username]\hfill\\ Tester's username.
\item[/DC020/ password]\hfill\\ Tester's password.

\paragraph{Example:}
\begin{verbatim}
<?xml version="1.0"?>
<telesales-buying-agent-login-data>
    <username>DeinBenutzername</username>
    <password>DeinPasswort</password>
</telesales-buying-agent-login-data>
\end{verbatim}
\end{description}
